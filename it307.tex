\documentclass[t,12pt]{beamer}
%
% Packages pour le français
\usepackage[T1]{fontenc} 
\usepackage[latin1]{inputenc}
\usepackage[frenchb]{babel}
\usepackage{color}

\definecolor{pblue}{rgb}{0.13,0.13,1}
\definecolor{pgreen}{rgb}{0,0.5,0}
\definecolor{pred}{rgb}{0.9,0,0}
\definecolor{pgrey}{rgb}{0.46,0.45,0.48}

\usepackage{listings}
\lstset{language=Java,
  showspaces=false,
  showtabs=false,
  breaklines=true,
  showstringspaces=false,
  breakatwhitespace=true,
  commentstyle=\color{pgreen},
  keywordstyle=\color{pblue},
  stringstyle=\color{pred},
  basicstyle=\small\ttfamily,
  columns=flexible,
  escapeinside=||,
  moredelim=[il][\textcolor{pgrey}]{$$},
  moredelim=[is][\textcolor{pgrey}]{\%\%}{\%\%}
}

%
% pour un pdf lisible à l'écran
% il y a d'autres choix possibles 
\usepackage{pslatex}
%
% pour le style et couleurs
\usetheme[secheader]{Boadilla}
\setbeamertemplate{itemize items}[circle]
\setbeamertemplate{enumerate items}[circle]
\setbeamertemplate{section in toc}[circle]
\setbeamertemplate{subsection in toc}[default]

%
% contenu de la page de titre
\title{Gestion de la Persistance}
\subtitle{IT307 - ENSEIRB MATMECA}
\author{Cyril Truchi}
\date{2016-2017}
%
% Fin du préambule
%

\begin{document}

\frame{\titlepage}

\section*{Sommaire}
\begin{frame}
	\tableofcontents
\end{frame}

\section{Introduction}

\begin{frame}
	\frametitle{Organisation du cours}

	Partage du cours avec Nicolas Philippe
\end{frame}

\begin{frame}
	\frametitle{Qui suis-je ?}
	
	\begin{itemize}
		\item Cyril Truchi
			\begin{itemize}
				\item D\'eveloppeur depuis 2004
				\item Master GL Bordeaux I en 2009
				\item Vacataire \`a l'Enseirb depuis 2016
			\end{itemize}
		\item Contact
			\begin{itemize}
				\item cyril.truchi@4sh.fr
			\end{itemize}
	\end{itemize}
\end{frame}

\begin{frame}
	\frametitle{Planning}

	\begin{itemize}
		\item Cyril
			\begin{description}
				\item [06 octobre : ] Introduction - Probl\'ematique de persistance d'un SI
				\item [20 octobre : ] Notions de persistance : Illustration avec JDBC
				\item [27 octobre : ] JPA avec hibernate - la base
				\item [03 novembre : ] JPA avec Hibernate - relation et hi\'erarchie
			\end{description}
		\item Nicolas
			\begin{description}
				\item [10 novembre : ] Introduction \`a NoSQL, Base cl\'e-valeur : Redis
				\item [24 novembre : ] Base document : MongoDB
				\item [08 d\'ecembre : ] Base graphe : Neo4j
				\item [15 d\'ecembre : ] Base colonne : Cassandra
			\end{description}
	\end{itemize}
\end{frame}



\section{Persistance}

\begin{frame}
	\tableofcontents[currentsection]
\end{frame}

\subsection{Cours}
\begin{frame}
	\frametitle{Sujet}

	Qu'entend-on par persistance ?
\end{frame}

\begin{frame}
	\frametitle{Couches JEE}

	\begin{itemize}
		\item Pr\'esentation
		\item Metier
		\item Persistance
	\end{itemize}
\end{frame}

\begin{frame}
	\frametitle{Data Access Object}

	\begin{block}{Wikipedia}
		In computer software, a data access object (DAO) is an object that provides an abstract interface to some type of database or other persistence mechanism. By mapping application calls to the persistence layer, DAO provide some specific data operations without exposing details of the database. This isolation supports the Single responsibility principle. It separates what data accesses the application needs, in terms of domain-specific objects and data types (the public interface of the DAO), from how these needs can be satisfied with a specific DBMS, database schema, etc. (the implementation of the DAO).
	\end{block}
	\url{https://en.wikipedia.org/wiki/Data\_access\_object}
\end{frame}

i\subsection{TP}
\begin{frame}
	\frametitle{TP}
	
	R\'ealiser la persistance d'une Company en m\'emoire.
\end{frame}

\subsection{Correction}
\begin{frame}
	\frametitle{Correction}

	\begin{itemize}
		\item Liste
		\item Map
	\end{itemize}
\end{frame}

\section{JDBC}

\begin{frame}
	\tableofcontents[currentsection]
\end{frame}

\subsection{Cours}

\begin{frame}
	\frametitle{JDBC}

	\begin{itemize}
		\item \underline{J}ava \underline{D}ata\underline{b}ase \underline{C}onnectivity
		\item Acc\`es bas niveau \`a la base de donn\'ees
	\end{itemize}
\end{frame}

\begin{frame}
	\frametitle{Relationnel : Avantages}

	\begin{itemize}
		\item Syst\`eme de requ\^etes tr\`es puissant
		\item Standardis\'ee (JDK 1.1)
		\item Excellente interop\'erabilit\'e
		\item Support transactionnelle en respectant les propri\'et\'es ACID
	\end{itemize}
\end{frame}

\begin{frame}
	\frametitle{ACID}

	\begin{description}
		\item \underline{\textbf{A}}tomicity : Tout ou rien
		\item \underline{\textbf{C}}onsistency : Respect des contraintes
		\item \underline{\textbf{I}}solation : Transactions ind\'ependantes
		\item \underline{\textbf{D}}urability : Conservation des donn\'ees
	\end{description}
\end{frame}

\begin{frame}
	\frametitle{Relationnel : Inconv\'enients}

	\begin{itemize}
		\item Scalabilit\'e limit\'ee
		\item Mapping Objet/Relationnel difficile
	\end{itemize}
\end{frame}

\begin{frame}
	\frametitle{JDBC : Concepts}

	\begin{itemize}
		\item Driver
		\item Connection/Datasource
		\item Statement/PreparedStatement
		\item ResultSet
	\end{itemize}
\end{frame}

\begin{frame}
	\frametitle{Driver}
\end{frame}

\begin{frame}
	\frametitle{Connection/Datasource}

	Datasource permet de d\'eleguer la cr\'eation des connections
\end{frame}

\begin{frame}
	\frametitle{Statement/PreparedStatement}

	\begin{enumerate}
		\item Parse the incoming SQL query 
		\item Compile the SQL query
		\item Plan/optimize the data acquisition path
		\setbeamercolor{item projected}{bg=red}
		\item Execute the optimized query / acquire and return data 
	\end{enumerate}
\end{frame}

\begin{frame}
	\frametitle{ResultSet}
\end{frame}

\begin{frame}[fragile]
	\frametitle{JDBC - Query}

	\begin{lstlisting}[frame=single]
Connection connection = dataSource.getConnection();
Statement statement = connection.createStatement();
ResultSet resultSet = statement.executeQuery("select * from MA_TABLE");

while (resultSet.next()) {
    String name = resultSet.getString("name");
    doSomethingWithName(name);
}
	\end{lstlisting}

	\begin{alertblock}{Attention}
		Il manque la gestion des erreurs, des transactions et la fermeture des ressources.
	\end{alertblock}
\end{frame}

\begin{frame}
	\frametitle{JDBC - Outil de plus haut niveau}
	
	\begin{itemize}
		\item Gestion des erreurs
		\item Gestion de la fermeture des ressources
		\item Concept de mapper
		\item Ex : 
			\begin{itemize}
				\item Spring jdbc template
				\item Apache Commons DB
				\item MyBatis
				\item jDBI
			\end{itemize}
	\end{itemize}
\end{frame}

\begin{frame}
	\frametitle{JDBI}

	\begin{itemize}
		\item Open Source
		\item Compact
		\item Facile d'acc\'es
		\item Actif
	\end{itemize}
\end{frame}

\begin{frame}[fragile]
	\frametitle{jDBI - Query}

	\begin{lstlisting}[frame=single]
private void doSomething() {
    try (Handle handle = dbi.open()) {
        List<MonObjet> mesObjets =
                handle.createQuery("SELECT * FROM MA_TABLE WHERE id = :monId")
                       .bind("monId", 1)
                       .map(MA_TABLE_MAPPER)
                       .list();
    }
}
	\end{lstlisting}
\end{frame}

\begin{frame}[fragile]
	\frametitle{jDBI - Mapper}

	\begin{lstlisting}[frame=single]
private static final ResultSetMapper<MonObjet> MA_TABLE_MAPPER =
        new ResultSetMapper<MonObjet>() {
            @Override
            public MonObjet map(int index,
                                ResultSet rs,
                                StatementContext ctx)
                    throws SQLException {
                MonObjet monObjet = new MonObjet();
                monObjet.setName(rs.getString("name"));
                return monObjet;
            }
        };
	\end{lstlisting}
\end{frame}

\begin{frame}[fragile]
	\frametitle{jDBI - Statements}

	\begin{lstlisting}[frame=single]
private void doSomething() {
    try (Handle handle = dbi.open()) {
        handle.createStatement("insert into MA_TABLE(name) values(:name)")
              .bind("name", "Cyril")
              .execute();
    }
}
	\end{lstlisting}
\end{frame}

\begin{frame}
	\frametitle{R\'ef\'erences}

	\begin{itemize}
		\item \url{http://jdbi.org/}
		\item \url{http://www.h2database.com}
	\end{itemize}
\end{frame}

\subsection{TP}

\begin{frame}
	\frametitle{TP}

	\begin{block}{R\'ealiser la persistance d'une Company avec jDBI sur la base H2 (base embarqu\'ee)}
		\begin{itemize}
			\item Impl\'ementer la classe \lstinline{JdbcCompanyRepository}
			\item Utiliser la propri\'et\'e \lstinline{dbi} inject\'ee via RestX
			\item La table est cr\'ee : regarder la classe \lstinline{DbiModule}
			\item \url{http://goo.gl/ekskDo}
			\item \lstinline{-Dpersistence=jdbc}
		\end{itemize}
	\end{block}
\end{frame}

\subsection{Correction}
\begin{frame}
	\frametitle{Correction}
	\begin{itemize}
		\item Liste
		\item Map
	\end{itemize}
\end{frame}

\section{JPA}

\begin{frame}
	\tableofcontents[currentsection]
\end{frame}

\subsection{Cours}

\begin{frame}
	\frametitle{Mapping Objet Relationnel}

	\begin{block}{Wikipedia}
		Un mapping objet-relationnel (en anglais object-relational mapping ou ORM) est une technique de programmation informatique qui cr\'ee l'illusion d'une base de donn\'ees orient\'ee objet \`a partir d'une base de donn\'ees relationnelle en d\'efinissant des correspondances entre cette base de donn\'ees et les objets du langage utilis\'e. On pourrait le d\'esigner par "correspondance entre monde objet et monde relationnel".
	\end{block}
	\url{https://fr.wikipedia.org/wiki/Mapping_objet-relationnel}

\end{frame}

\begin{frame}
	\frametitle{Contraintes sur le mod\`ele objet}

	\begin{alertblock}{Obligatoire}
		\begin{itemize}
			\item Constructeur sans argument
		\end{itemize}
	\end{alertblock}

	\begin{block}{Recommand\'e}
		\begin{itemize}
			\item Acc\`es aux propri\'et\'es mapp\'ees par getter/setter
			\item Classes non finales
			\item Propri\'et\'e servant d'identifiant
		\end{itemize}
	\end{block}
\end{frame}

\begin{frame}[fragile]
	\frametitle{Exemple}

	\begin{lstlisting}[frame=single, basicstyle=\tiny]
public class Event {

    private Long id;
    |\pause|private String title;
    private Contact contact;

    |\pause|public Event() {
    }

    |\pause|public Long getId() {
        return id;
    }
    private void setId(Long id) {
        this.id = id;
    }
    public String getTitle() {
        return title;
    }
    public void setTitle(String title) {
        this.title = title;
    }
    public Contact getContact() {
        return contact;
    }
    public void setContact(Contact contact) {
        this.contact = contact;
    }
}
	\end{lstlisting}
\end{frame}

\begin{frame}
	\frametitle{D\'eclaration du mapping}

	\begin{itemize}
		\item Fichier de mapping (hbm.xml ou orm.xml)
			\begin{itemize}
				\item Isolation mapping et classe
			\end{itemize}
		\item Annotations
			\begin{itemize}
				\item Facile d'usage
				\item Maintenance plus facile
			\end{itemize}
	\end{itemize}
\end{frame}

\begin{frame}[fragile]
	\frametitle{Exemple avec annotations}

	\begin{lstlisting}[frame=single]
import javax.persistence.Entity;
import javax.persistence.Id;

@Entity
public class Event {

    @Id
    private Long id;
    private String title;
    private Contact contact;
}
	\end{lstlisting}
\end{frame}

\begin{frame}[fragile]
	\frametitle{Champ embarqu\'e}

	\begin{lstlisting}[frame=single]
import javax.persistence.Entity;
import javax.persistence.Id;

@Entity
public class Event {

    @Id
    private Long id;
    private String title;
    @Embedded
    private Contact contact;

    ...
}
	\end{lstlisting}
\end{frame}

\begin{frame}[fragile]
	\frametitle{Champ embarqu\'e}

	\begin{lstlisting}[frame=single]
import javax.persistence.Embeddable;

@Embeddable
public class Contact {

    private String firstName;
    private String lastName;
}
	\end{lstlisting}
\end{frame}

\begin{frame}[fragile]
	\frametitle{EntityManager}

	\begin{itemize}
		\item Point d'acc\'s sur la base mapp\'ee
		\item Inject\'e dans les composants JavaEE
	\end{itemize}

	\begin{lstlisting}[frame=none]
    @PersistenceContext
    private EntityManager em;
	\end{lstlisting}
\end{frame}

\begin{frame}[fragile]
	\frametitle{EntityManager}

	\begin{itemize}
		\item Hors JavaEE
	\end{itemize}

	\begin{lstlisting}[frame=none]
    EntityManagerFactory emf = ...;
    EntityManager em = emf.createEntityManager();

    try {
        em.getTransaction().begin();
        doSomethingWithEm(em);
    } finally {
        em.getTransaction().commit();
        em.close();
    }
	\end{lstlisting}
\end{frame}

\begin{frame}[fragile]
	\frametitle{Requ\^etes}

	\begin{block}{JPQL}
		\begin{lstlisting}[frame=none]
em.createQuery("select c from Company c", Company.class)
  .getResultList()
		\end{lstlisting}
	\end{block}

	\begin{exampleblock}{Alternative}
		\begin{lstlisting}[frame=none]
em.createQuery("from Company c", Company.class)
  .getResultList()
		\end{lstlisting}
	\end{exampleblock}
\end{frame}

\begin{frame}[fragile]
	\frametitle{Requ\^ete par id}

	\begin{lstlisting}[frame=none]
    em.find(Event.class, id);
	\end{lstlisting}
\end{frame}

\begin{frame}
	\frametitle{Stockage}

	\begin{exampleblock}{Cr\'eation}
		\lstinline{em.create(event);}
	\end{exampleblock}

	\begin{block}{Modification}
		\lstinline{em.merge(event);}
	\end{block}

	\begin{alertblock}{Suppression}
		\lstinline{em.delete(event);}
	\end{alertblock}
\end{frame}

\begin{frame}[fragile]
	\frametitle{G\'en\'eration d'Id}

	\begin{block}{Automatique}
		\begin{lstlisting}[frame=none]
@Id
@GeneratedValue(strategy = AUTO)
private long id;
		\end{lstlisting}
	\end{block}

	\begin{alertblock}{Attention}
		D\'ependant de la base de donn\'ees.

		Pour une persistance polyglotte, on pr\'ef\'erera les UUID.
	\end{alertblock}
\end{frame}

\subsection{TP}

\begin{frame}
	\frametitle{TP}

	\begin{block}{R\'ealiser la persistance d'une Company avec Hibernate (JPA) sur la base H2 (base embarqu\'ee)}
		\begin{itemize}
			\item Annoter la class \lstinline{Company}
			\item Impl\'ementer la classe \lstinline{JpaCompanyRepository}
			\item Utiliser la propri\'et\'e \lstinline{emf} inject\'ee via RestX
			\item La table est cr\'ee automatiquement par hibernate
			\item \url{http://goo.gl/ekskDo}
			\item \lstinline{-Dpersistence=jpa}
		\end{itemize}
	\end{block}
\end{frame}

\end{document}
